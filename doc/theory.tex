% !TeX program = xelatex
% !TeX spellcheck = ru_RU
\documentclass[DIV=11,12pt,a4paper]{scrartcl}
\usepackage[colorlinks,linkcolor=black,citecolor=blue,urlcolor=blue]{hyperref}
\usepackage[russian]{babel}
\usepackage{amssymb}
\usepackage{amsmath}
\usepackage{graphicx}
\usepackage{physics}
\usepackage{microtype}

\usepackage{unicode-math}
\setmainfont{STIX Two Text}
\setmathfont{STIX Two Math}
\setsansfont[Scale=0.99]{PT Sans}
\linespread{1.1}
\KOMAoptions{DIV=last}

%\frenchspacing
\begin{document}
\tableofcontents

\section{Базовые уравнения}

Рассматривается задача о движении и излучении ленточного пучка электронов в планарном гофрированном волноводе.
Предполагается, что излучение сконцентрировано в TEM-моде волновода.
Уравнение возбуждения такой моды записывается как
\begin{equation}
    \frac{1}{c^2}\pdv[2]{A}{t} - \pdv[2]{A}{z} = \frac{4\pi}{c} \bar{h} l_0 \cos(\bar{h} z) J(t,z),
    \label{eq:Adimension}
\end{equation}
где $A=A_x$~--- поперечный вектор-потенциал в TEM-моде, $\bar{h} = 2\pi/d$, $d$~--- период гофрировки волновода, $l_0(z)$~--- глубина гофрировки волновода, $b$ --- расстояние между пластинами волновода, $J = b^{-1}\int_{-b/2}^{b/2} j_z(t,x,z) \dd{x}$.

В предположении, что электроны движутся только по продольной координате, на них действует продольная сила
\begin{equation}
    F_z = -e E \cos(\bar{h}z) \bar{h} l_0,
\end{equation}
где 
\begin{equation}
    E = E_x = -\frac{1}{c}\pdv{A}{t}.
\end{equation}

Перейдем к безразмерным переменным.
Наличие периода гофрировки $d$ задает естественный пространственный размер, поэтому введем $\vb{r}' = \bar{h} \vb{r}$, $t' = c\bar{h} t$, $A' = eA/mc^2$, $E' = eE/mc^2\bar{h}$, $j_z' = j_z / c \bar{n}$, где $\bar{n}$ --- критическая концентрация для частоты $c\bar{h}$.
В таком случае $J' = J/c \bar{n}$.
Тогда уравнение \eqref{eq:Adimension} запишется в виде
\begin{equation}
    \pdv[2]{A}{t} - \pdv[2]{A}{z} = J(t, z) l_0(z) \cos z.
\end{equation}
Штрихи в этом уравнении опущены.
Вводя безразмерный импульс $p' = p/mc$, получим для движения электрона уравнение
\begin{equation}
    \dv{p}{t} = - E l_0 \cos z, \quad \dv{z}{t} = \frac{p}{\gamma}.
\end{equation}
Здесь штрихи также опущены.

\end{document}