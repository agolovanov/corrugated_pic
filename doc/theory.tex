% !TeX program = xelatex
% !TeX spellcheck = ru_RU
\documentclass[DIV=11,12pt,a4paper]{scrartcl}
\usepackage[colorlinks,linkcolor=black,citecolor=blue,urlcolor=blue]{hyperref}
\usepackage[russian]{babel}
\usepackage{amssymb}
\usepackage{amsmath}
\usepackage{graphicx}
\usepackage{physics}
\usepackage{microtype}

\usepackage{unicode-math}
\setmainfont{STIX Two Text}
\setmathfont{STIX Two Math}
\setsansfont[Scale=0.99]{PT Sans}
\linespread{1.1}
\KOMAoptions{DIV=last}

%\frenchspacing
\begin{document}
\tableofcontents

\section{Базовые уравнения}

Рассматривается задача о движении и излучении ленточного пучка электронов в планарном гофрированном волноводе.
Предполагается, что излучение сконцентрировано в TEM-моде волновода.
Уравнение возбуждения такой моды записывается как
\begin{equation}
    \frac{1}{c^2}\pdv[2]{A}{t} - \pdv[2]{A}{z} = \frac{4\pi}{c} \bar{h} l_0 \cos(\bar{h} z - \bar{h} z_0) J(t,z),
    \label{eq:Adimension}
\end{equation}
где $A=A_x$~--- поперечный вектор-потенциал в TEM-моде, $\bar{h} = 2\pi/d$, $d$~--- период гофрировки волновода, $l_0(z)$~--- глубина гофрировки волновода, $b$ --- расстояние между пластинами волновода, $J = b^{-1}\int_{-b/2}^{b/2} j_z(t,x,z) \dd{x}$.

В предположении, что электроны движутся только по продольной координате, на них действует продольная сила
\begin{equation}
    F_z = -e E \cos(\bar{h}z - \bar{h}z_0) \bar{h} l_0,
\end{equation}
где 
\begin{equation}
    E = E_x = -\frac{1}{c}\pdv{A}{t}.
\end{equation}

Перейдем к безразмерным переменным.
Наличие периода гофрировки $d$ задает естественный пространственный размер, поэтому введем $\vb{r}' = \bar{h} \vb{r}$, $t' = c\bar{h} t$, $A' = eA/mc^2$, $E' = eE/mc^2\bar{h}$, $j_z' = j_z / c \bar{n}$, где $\bar{n}$ --- критическая концентрация для частоты $c\bar{h}$.
В таком случае $J' = J/c \bar{n}$.
Тогда уравнение \eqref{eq:Adimension} запишется в виде
\begin{equation}
    \pdv[2]{A}{t} - \pdv[2]{A}{z} = J(t, z) l_0(z) \cos (z-z_0).
\end{equation}
Штрихи в этом уравнении опущены.
Вводя безразмерный импульс $p' = p/mc$, получим для движения электрона уравнение
\begin{equation}
    \dv{p}{t} = - E l_0 \cos (z-z_0), \quad \dv{z}{t} = \frac{p}{\gamma}.
\end{equation}
Здесь штрихи также опущены.

\section{Представление макрочастиц}

Будем считать, что пучок электронов представляет собой множество макрочастиц
\begin{equation}
    J(t,z) = -\sum_i Q_i v_i(t) S(z - z_i(t)),
\end{equation}
где $Q_i > 0$~--- весовые коэффициенты макрочастиц, $v_i = p_i / \gamma_i$~--- скорости макрочастиц, $S(z)$~--- функция, описывающая форму макрочастицы.
Тогда система уравнений, которую нам требуется решать, запишется как
\begin{align}
    &\pdv{A}{t} = - E,\\
    &\pdv{E}{t} = -\pdv[2]{A}{z} + \sum_i Q_i v_i S(z - z_i) l_0(z) \cos (z-z_0),\\
    &\dv{p_i}{t} = -\int_{-\infty}^{\infty} S(z-z_i(t)) E(z,t) l_0(z)  \cos (z-z_0) \dd{z},\\
    &\dv{z_i}{t} = v_i.
\end{align}
Здесь полагается, что $\int_{-\infty}^\infty S(z) \dd{z} = 1$, а $p_i$ является импульсом каждого электрона в макрочастице.

\section{Схема решения уравнений для поля}

Рассмотрим систему уравнений 
\begin{align}
    &\pdv{A}{t} = -E,\\
    &\pdv{E}{t} = -\pdv[2]{A}{z} + S.
\end{align}
Введем пространственную сетку с шагом $\Delta x$ и преобразуем уравнения к конечным разностям:
\begin{align}
    &\pdv{A_i}{t} = -E_i,\\
    &\pdv{E_i}{t} = - \frac{A_{i+1} - 2 A_i + A_{i-1}}{\Delta z^2} + S_i.
\end{align}
Для интегрирования по времени используем leapfrog-схему, в которой $E_i$ и $A_i$ берутся в разные моменты времени
\begin{align}
    &\frac{A_i^{n+1/2} - A_i^{n-1/2}}{\Delta t} = - E_i^n,\\
    &\frac{E_i^{n+1} - E_i^{n}}{\Delta t} = - \frac{A_{i+1}^{n+1/2} - 2 A_i^{n+1/2} + A_{i-1}^{n+1/2}}{\Delta z^2} + S_i^n.
\end{align}

\subsection{Граничные условия}
На границах области ставится граничное условие излучения.
На левой и правой границе, соответственно,
\begin{equation}
    \pdv{A}{t} - \pdv{A}{z} = 0, \quad \pdv{A}{t} + \pdv{A}{z} = 0.
\end{equation}
Рассмотрим левую границу и используем на ней разностную схему первого порядка
\begin{equation}
    \frac{A_0^{n+3/2} - A_0^{n+1/2}}{\Delta t} - \frac{A_1^{n+1/2} - A_0^{n+1/2}}{\Delta z} = 0.
\end{equation}
Разность в числителе первой дроби в точности равна $- E_0^{n+1} \Delta t$ в нашей численной схеме.
Таким образом, граничные условия:
\begin{equation}
    E_0^{n+1} = \frac{A_0^{n+1/2} - A_1^{n+1}}{\Delta z}, \quad E_{N-1}^{n+1} = \frac{A_{N-1}^{n+1/2} - A_{N-2}^{n+1}}{\Delta z}.
\end{equation}

\end{document}